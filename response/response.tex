\documentclass[12pt]{article}

\usepackage[margin=1in]{geometry}
\usepackage{natbib}
\usepackage{listings}
\usepackage{rotating, graphicx}
\usepackage{booktabs, natbib}
\usepackage[hyphens]{url}
% \usepackage [english]{babel}
\usepackage{amsmath, amsbsy, amsthm, epsfig, epsf, psfrag, graphicx,
  amssymb, enumerate, bm}
\usepackage{enumitem}

\usepackage{color}

\newcommand{\jy}[1]{\textcolor{red}{JY #1}}
\newcommand{\blue}[1]{\textcolor{blue}{#1}}

\usepackage[colorlinks=true, urlcolor=blue, linkcolor=blue, citecolor=blue]{hyperref}




\newenvironment{comment}%
{\begin{quotation}\noindent\small\it\ignorespaces%
  }{\end{quotation}}



\begin{document}

\begin{center}
  {\Large\bf Response Letter to Reviewers' Comments}
\end{center}


\section{Summary}

I would like to that Lucy Liu for her thoughtful, constructive
comments. The manuscript has been revised accordingly with the
following major changes:

\begin{enumerate}
\item
  Following the reviewer's comments on expanding the literature
  review, a new paragraph was added to the introduction that
  expanded on other works that have been done. This involved reviewing
  alternatives to the plus-minus and works that other people
  within the sports statistics world have done to address this.
  This also included addressing some downsides to these works and
  why this project is necessary.
  
\item
  In response to the reviewer's comment on clearly stating the 
  hypothesis, the hypothesis was added to the last paragraph of the 
  introduction that asked the over-arching question that is being
  addressed in the paper. This set up the work that is being done
  in the paper and what the readers can hope to take from this work.

\item
  The reasons that each of the models used were expanded. The need
  for the assumptions to varified was added to the methods section.
  The reviewer did make the suggestion of adding the plots to varify 
  that the assumptions were satisfied. While I did make plots to make
  sure all assumptions were varified, they were not included in the final
  paper due to space concerns. Also, I did not want to take attention
  away from the discussion at hand. There were additions made to the 
  ridge regression section explaining why ridge was chosen over other 
  regularization techniques.

\item 
  An extra paragraph was added to the end of the discussion to explain
  future directions. The reason for this study was re-iterated as well as
  explaining how it can be used in the NHL. Additional comments were added
  to explain what can be done in the future based on this study and the
  work of other people within the field.

\end{enumerate}


Point-by-point responses are as follows, with the reviewer/editor's comments in {\it italic}.


\section{Response to Each Comment}

\begin{comment}
  The literature review could be expanded to include additional studies that 
  have examined the plus-minus statistic and its alternatives. This could provide 
  a better foundation for the reasoning behind this study and highlight gaps
  that you hope to fill with your current research.
\end{comment}

This was a good note to improve the paper. While I initially mentioned Brian MacDonald's 
work and impliedthat other people had proposed alternatives to the plus-minus, I did not 
include enough detail. Including more literature helps the reader to have a better understanding
of what work is already being done. It also allows me to explain why my study is 
important and what it contributed to this conversation. A paragraph better explaining 
the literature review was added as a fourth paragraph in the introduction.


\begin{comment}
  The paper evaluates the effectiveness of the plus-minus statistic compared 
  to other metrics like Corsi and Fenwick. While the objectives are clear, the 
  hypotheses are not explicitly stated. Clearly state the hypotheses at the end 
  of the introduction.
\end{comment}

In the first draft, the overall goal of the project was stated as well as the 
questions that would be answered in the paper. At the request of the reviewer,
the hypothesis of the project was stated in the second to last sentence of the
introduction. Hopefully this helps the readers to understand the over-arching 
question that is being answered in this project.

\begin{comment}
  Although each method is described individually, the reason behind selecting these 
  specific methods (Ex.why ridge regression and not another regularization technique) 
  is not fully explained. Add a brief justification for each method and explain why 
  alternatives were not selected.

  Additionally, clearly outline the assumptions associated with each statistical 
  method and discuss how these assumptions are checked and addressed within the 
  analysis. Include diagnostic tests and results that assess whether the assumptions 
  of each statistical method are satisfied. For example, provide residual plots 
  or normality tests for mixed-effects models.
\end{comment}



\begin{comment}
   The discussion section is missing future directions. Discuss what comes next after 
   this study, or some possible new ideas or improvements to explore. Further elaborate 
   on the implications of these findings for the NHL.
\end{comment}


\bibliographystyle{chicago}
\bibliography{growth}

%\atColsBreak{\pagediscards}
\end{document}