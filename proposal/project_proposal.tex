\documentclass[12pt]{article}

%% preamble: Keep it clean; only include those you need
\usepackage{amsmath}
\usepackage[margin = 1in]{geometry}
\usepackage{graphicx}
\usepackage{booktabs}
\usepackage{natbib}

% for space filling
\usepackage{lipsum}
% highlighting hyper links
\usepackage[colorlinks=true, citecolor=blue]{hyperref}


%% meta data

\title{Proposal: An Analysis of Hockey's Most Controversial Statistic}
\author{Melanie Desroches\\
  Department of Statistics\\
  University of Connecticut
}

\begin{document}
\maketitle


\paragraph{Introduction}
In hockey, a very common yet controversial statistic that is reported is the plus/minus statistic. Plus/minus is 
calculated by taking all the even strength goals that were scored by their team while they were on the ice and 
subtracting it by the total number of goals that were scored by the other team while they were on the ice. The 
idea behind the number is that it will show if a player has an overall positive or negative contribution when 
they are playing. However, the plus/minus has a lot of drawbacks. For example, in ice hockey, players tend to play 
on "lines", meaning the same three forwards and the same two defencemen tend to play at the same time. Because of this, 
plus/minus is heavily influenced by who plays with who. Because of this, some argue that the plus/minus is more of a team 
statistic rather than an individual statistic. Many of the criticisms of this statistic comes from the fact that
there are a lot of confounding variables that influence how the plus/minus is calculated. Alternatives to the plus/minus 
statistic have been developed, such as COSRI or Fenwick. Understanding the limitations and potential of plus/minus can 
help coaches, analysts, and fans better interpret player data. While advanced statistics are valuable, plus/minus is still 
popular, especially among casual fans. The project explores whether the plus/minus still holds relevance or whether it should 
be supplemented or replaced with other metrics.


\paragraph{Specific Aims}
Formulate your research question;
translate your research question into statistical/data science questions

The goal of this project is to evaluate the effectiveness of the plus/minus statistic. Some of the information
I am trying to determine is if the plus/minus statistic can be used to determine how good a player is both on
offense and defense. I also want to see if the plus/minus has any validity being used as an individual
statistic or as a team statistic. Based on my observations, I will compare the plus/minus statistic to other
statistics that have a similar purpose such as CORSI or Fenwick.
These questions are critical as plus/minus is used in scouting and analysis, but its reliability 
has been debated. By comparing it with modern metrics, I aim to shed light on its true value in player evaluation.



\paragraph{Data}

I am using data from the National Hockey League (NHL). The data is collected from a website called Natural Stat Trick. 
I collected data from the 2021-2022, 2022-2023, and 2023-2024 seasons. I am particularly interested in the advanced hockey 
statistics that are collected for each player in the NHL. These statistics include CORSI, Fenwick, high-danger chances,
shooting percentages, and more.

\paragraph{Research Design and Methods}
What design or methods will you use?

Since the plus/minus statistic tends to deal with collinearity, a ridge regression might be more
appropriate for analysis. Similar work has been done in \cite{Macdonald_2012}


\paragraph{Discussion}
What are the most challenge parts of the task?
What are the limitations of your work? What is your fall-back plan if
something unexpected happens?


\bibliography{../manuscript/refs}
\bibliographystyle{chicago}

\end{document}