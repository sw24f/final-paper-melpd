\documentclass[12pt]{article}

%% preamble: Keep it clean; only include those you need
\usepackage{amsmath}
\usepackage[margin = 1in]{geometry}
\usepackage{graphicx}
\usepackage{booktabs}
\usepackage{natbib}

% for space filling
\usepackage{lipsum}
% highlighting hyper links
\usepackage[colorlinks=true, citecolor=blue]{hyperref}


%% meta data

\title{Proposal: An Analysis of Hockey's Most Controversial Statistic}
\author{Melanie Desroches\\
  Department of Statistics\\
  University of Connecticut
}

\begin{document}
\maketitle


\paragraph{Introduction}
In ice hockey, a very common yet controversial statistic that is reported is the plus/minus statistic. It measures a player's impact 
by calculating the difference between the even strength goals their team scores and concedes while they are on the ice. This metric
aims to reflect whether a player contributes positively or negatively to the game However, the plus/minus has significant of drawbacks. 
According to Brian MacDonald "The traditional plus-minus statistic in hockey is highly dependent on a player’s teammates and opponents" 
\cite{Macdonald_2011}. Hockey players typically play in fixed "lines" where the same group of forwards and defensemen are on the ice 
together. This makes plus/minus heavily influenced by a player's linemates rather than their individual performance. As a result, many
argue that the statistic functions more as a reflection of team dynamics than as a precise indicator of individual contribution. 
Furthermore, numerous confounding variables, such as the quality of the opponent and situational factors, make the calculation less 
reliable. Alternatives to the plus/minus statistic have been developed, such as Corsi or Fenwick. Understanding the limitations and
potential of plus/minus can help coaches, analysts, and fans better interpret player data. While advanced statistics are valuable, 
plus/minus is still popular, especially among casual fans. This project explores whether the plus/minus still holds relevance or 
whether it should be supplemented or replaced with other metrics.


\paragraph{Specific Aims}
The goal of this project is to evaluate the effectiveness of the plus/minus statistic. One of the topics of discussion is if the 
plus/minus statistic can be used to determine how good a player is both on offense and defense. I also want to see if the plus/minus 
has any validity being used as an individual statistic or as a team statistic. Based on my observations, I will compare the plus/minus 
statistic to other statistics that have a similar purpose such as CORSI or Fenwick. These questions are critical as plus/minus is used 
in scouting and analysis, but its reliability has been debated. By comparing it with modern metrics, I aim to shed light on its true 
value in player evaluation.



\paragraph{Data}

I am utilizing data from the National Hockey League (NHL), sourced from websites like Natural Stat Trick and NHLEdge.
The data will be collected from the 2021-2022, 2022-2023, and 2023-2024 seasons. I am particularly interested in the advanced hockey 
statistics that are collected for each player in the NHL. These advanced metrics include Corsi, Fenwick, high-danger chances, shooting 
percentages, and other performance indicators. Additionally, the dataset features traditional statistics such as points, 
goals, assists, plus/minus, and time on ice, providing a comprehensive view of player performance across multiple seasons.



\paragraph{Research Design and Methods}
This study will evaluate the effectiveness of the plus/minus statistic in ice hockey to determine whether it can assess a 
player's offensive or defensive contributions and if it should be treated as an individual or team statistic. The analysis 
will also compare plus/minus with other advanced hockey metrics, such as Corsi and Fenwick, to assess its overall utility.
The first step of the project is to collect all relevant data. In this case, I will be exporting the data from all players
from the 2021- 2024 seasons into an Excel or csv file.  Following data collection, the next phase will involve data cleaning to 
ensure accuracy and consistency. Once the dataset is prepared, initial analyses will identify correlations among variables 
and pinpoint issues of collinearity, which could impact the selection of models and variables for further analysis.
To evaluate how plus/minus can be employed to assess offensive and defensive contributions, I will implement ridge regression, 
which is well-suited for handling collinearity concerns inherent in the plus/minus statistic, as 
noted in prior research in \cite{Macdonald_2012}. The coefficients from the ridge regression model will reveal 
whether offensive or defensive metrics contribute more to explaining plus/minus. I plan on using a similar method for comparing
the plus/minus with other advanced statistics, replacing the offensive and defensive statistics with CORSI, Fenwick, and any other
advanced statistics that are deemed relevant. When assessing if plus/minus is more influenced by individual performance or team 
performance, I am considering using a mixed-effects ridge regression model. By using ridge regression with mixed-effects, it will 
control for the potential influence of team quality on individual plus/minus values. This will help us determine how much of the 
variability in plus/minus is explained by the team versus individual players.



\paragraph{Discussion}
The most challenging aspect of this project will be addressing collinearity, as many of the statistics involved are highly correlated. 
To mitigate this issue, I plan to implement ridge regression models. According to Brian MacDonald in his ridge regression analysis 
of the plus/minus, "Ridge regression frequently reduces the error bounds in the estimates and improves the predictive performance 
of the model when collinearly exists in the data. Ridge regression introduces bias in the estimates, but the tradeoff is typically 
worthwhile." \cite{Macdonald_2012}. Mixed-effects models, principal component analysis, or other models could be used if ridge regression 
proves to be ineffective. I am also assuming that the data that I am collecting from Natural Stat Trick is reliable and accurate. While 
Natural Stat Trick provides advanced metrics, the quality and completeness of the data may vary.  Therefore, I will be particularly diligent 
in the data cleaning process to ensure that the variables of interest are available and consistent throughout the dataset. This 
careful attention to detail will help enhance the resilience and reliability of the analysis and the validity of the results.


\bibliography{../manuscript/refs}
\bibliographystyle{chicago}

\end{document}