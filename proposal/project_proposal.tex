\documentclass[12pt]{article}

%% preamble: Keep it clean; only include those you need
\usepackage{amsmath}
\usepackage[margin = 1in]{geometry}
\usepackage{graphicx}
\usepackage{booktabs}
\usepackage{natbib}

% for space filling
\usepackage{lipsum}
% highlighting hyper links
\usepackage[colorlinks=true, citecolor=blue]{hyperref}


%% meta data

\title{Proposal: An Analysis of Hockey's Most Controversial Statistic}
\author{Melanie Desroches\\
  Department of Statistics\\
  University of Connecticut
}

\begin{document}
\maketitle


\paragraph{Introduction}
In hockey, a very common yet controversial statistic that is reported is the plus/minus statistic. Plus/minus is 
calculated by taking all the even strength goals that were scored by their team while they were on the ice and 
subtracting it by the total number of goals that were scored by the other team while they were on the ice. The 
idea behind the number is that it will show if a player has an overall positive or negative contribution when 
they are playing. However, the plus/minus has a lot of drawbacks. For example, in ice hockey, players tend to play 
on "lines", meaning the same three forwards and the same two defencemen tend to play at the same time. Because of this, 
plus/minus is heavily influenced by who plays with who. Because of this, some argue that the plus/minus is more of a team 
statistic rather than an individual statistic. Many of the criticisms of this statistic comes from the fact that
there are a lot of confounding variables that influence how the plus/minus is calculated. According to Brian MacDonald 
"The traditional plus-minus statistic in hockey is highly dependent on a player’s teammates and opponents" \cite{Macdonald_2011}.
Alternatives to the plus/minus statistic have been developed, such as COSRI or Fenwick. Understanding the limitations and
potential of plus/minus can help coaches, analysts, and fans better interpret player data. While advanced statistics are valuable, 
plus/minus is still popular, especially among casual fans. The project explores whether the plus/minus still holds relevance or 
whether it should be supplemented or replaced with other metrics.


\paragraph{Specific Aims}
The goal of this project is to evaluate the effectiveness of the plus/minus statistic. Some of the information I am trying
to determine is if the plus/minus statistic can be used to determine how good a player is both on offense and defense. I 
also want to see if the plus/minus has any validity being used as an individual statistic or as a team statistic. Based on my 
observations, I will compare the plus/minus statistic to other statistics that have a similar purpose such as CORSI or Fenwick. 
These questions are critical as plus/minus is used in scouting and analysis, but its reliability has been debated. By comparing 
it with modern metrics, I aim to shed light on its true value in player evaluation.



\paragraph{Data}

I am using data from the National Hockey League (NHL). The data is collected from the websites Natural Stat Trick and NHLEdge. 
I collected data from the 2021-2022, 2022-2023, and 2023-2024 seasons. I am particularly interested in the advanced hockey 
statistics that are collected for each player in the NHL. These statistics include CORSI, Fenwick, high-danger chances,
shooting percentages, and more. This data will also include more basic statistics such as points, goals, assists, plus/minus,
time on ice, etc.


\paragraph{Research Design and Methods}
This study will evaluate the effectiveness of the plus/minus statistic in ice hockey to determine whether it can assess a 
player's offensive or defensive contributions and if it should be treated as an individual or team statistic. The analysis 
will also compare plus/minus with other advanced hockey metrics, such as Corsi and Fenwick, to assess its overall utility.

The first step of the project is to collect all relevant data. In this case, I will be exporting the data from all players
from the 2021- 2024 seasons into an Excel or csv file. Once the data has been collected, it needs to be cleaned. After the
data has been prepared, I can start some initial analysis to see what variables are correlated and where issues of collinearity 
exist, since this will effect which models and variables I use. 

In order to determine if or how plus/minus can be used to assess offensive and defensive contribution, I will use a ridge 
regression. Since the plus/minus statistic tends to deal with collinearity, a ridge regression might be more appropriate for 
analysis. Similar work has been done in \cite{Macdonald_2012}. The coefficients from the ridge regression model will reveal 
whether offensive or defensive metrics contribute more to explaining plus/minus. I plan on using a similar 



\paragraph{Discussion}
What are the most challenge parts of the task?
What are the limitations of your work? What is your fall-back plan if
something unexpected happens?

According to Brian MacDonald in his ridge regression analysis of the plus/minus, "Ridge regression frequently reduces the error 
bounds in the estimates and improves the predictive performance of the model when collinearly exists in the data. Ridge regression 
introduces bias in the estimates, but the tradeoff is typically worthwhile." \cite{Macdonald_2012}.

\bibliography{../manuscript/refs}
\bibliographystyle{chicago}

\end{document}