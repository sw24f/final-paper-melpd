\documentclass[12pt]{article}

%% preamble: Keep it clean; only include those you need
\usepackage{amsmath}
\usepackage[margin = 1in]{geometry}
\usepackage{graphicx}
\usepackage{booktabs}
\usepackage{natbib}

% for space filling
\usepackage{lipsum}
% highlighting hyper links
\usepackage[colorlinks=true, citecolor=blue]{hyperref}


%% meta data

\title{Proposal: An Analysis of Hockey's Most Controversial Statistic}
\author{Melanie Desroches\\
  Department of Statistics\\
  University of Connecticut
}

\begin{document}
\maketitle


\paragraph{Introduction}
In hockey, a very common statistic that is reported is the plus/minus statistic. Plus/minus is calculated
by taking all the even strength goals that were scored by their team was on the ice and subtracting it by 
the total number of goals that were scored by the other team. The idea behind the number is that it will 
show if a player has an overall positive or negative contribution when they are playing. However, the plus/minus
has a lot of drawbacks. For example, in ice hockey, players tend to play on "lines", meaning the same three 
forwards and the same two defencemen tend to play together. Because of this, plus/minus is heavily influenced 
by who they play with. Some would argue that the plus/minus is more of a team statistic rather than an individual 
statistics. Many of the criticisms of this statistic comes from the fact that there are a lot of confounding 
variables that influence how the plus/minus is calculated. 

The goal of this paper is to evaluate the effectiveness of the plus/minus statistic. Some of the information
I am trying to determine is if the plus/minus statistic can be used to determine how good a player is both on
offense and defense. I also want to see if the plus/minus has any validity being used as an individual
statistic or as a team statistic. Based on my observations, I will compare the plus/minus statistic to other
statistics that have a similar purpose such as CORSI or Fenwick.

\paragraph{Specific Aims}
Formulate your research question;
translate your research question into statistical/data science questions

\lipsum[2]

\paragraph{Data}
Hopefully, you have identified the data needed for your project. Give a
description about it.

\lipsum[3]

\paragraph{Research Design and Methods}
What design or methods will you use?


\lipsum[4]

\paragraph{Discussion}
What are the most challenge parts of the task?
What are the limitations of your work? What is your fall-back plan if
something unexpected happens?

\lipsum[5]

\bibliography{../manuscript/refs}
\bibliographystyle{chicago}

\end{document}