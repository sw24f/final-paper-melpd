\documentclass[12pt]{article}

%% preamble: Keep it clean; only include those you need
\usepackage{amsmath}
\usepackage[margin = 1in]{geometry}
\usepackage{graphicx}
\usepackage{booktabs}
\usepackage{natbib}

%% for double spacing
\usepackage{setspace}

% for space filling
\usepackage{lipsum}
% highlighting hyper links
\usepackage[colorlinks=true, citecolor=blue]{hyperref}

% Get line numbers for ease of referencing for the reviewers
\usepackage[]{lineno}
\linenumbers*[1]
% patches to make lineno work better with amsmath
\newcommand*\patchAmsMathEnvironmentForLineno[1]{%
        \expandafter\let\csname old#1\expandafter\endcsname\csname 
        #1\endcsname
        \expandafter\let\csname oldend#1\expandafter\endcsname\csname 
        end#1\endcsname
        \renewenvironment{#1}%
        {\linenomath\csname old#1\endcsname}%
        {\csname oldend#1\endcsname\endlinenomath}}%
\newcommand*\patchBothAmsMathEnvironmentsForLineno[1]{%
        \patchAmsMathEnvironmentForLineno{#1}%
        \patchAmsMathEnvironmentForLineno{#1*}}%
\AtBeginDocument{%
        \patchBothAmsMathEnvironmentsForLineno{equation}%
        \patchBothAmsMathEnvironmentsForLineno{align}%
        \patchBothAmsMathEnvironmentsForLineno{flalign}%
        \patchBothAmsMathEnvironmentsForLineno{alignat}%
        \patchBothAmsMathEnvironmentsForLineno{gather}%
        \patchBothAmsMathEnvironmentsForLineno{multline}%
}



%% meta data

\title{Hockey's Most Controversial Statistic: An Analysis of the Plus-Minus Statistic}
\author{Melanie Desroches\\
  Department of Statistics\\
  University of Connecticut
}

\begin{document}
\maketitle

\begin{abstract}
Here is the abstract.  
\end{abstract}

\doublespacing

\section{Introduction}
\label{sec:intro}

Use this section to answer three questions:
Why is the topic important/interesting?
What has been done on this topic in the literature?
What is your contribution?

Within the realm of sports, many different statistics are used to determine how good a player is. This can include points,
assists, time played, etc. But is there just one valaue that can be used to determine the effectiveness of a player? This is an 
important topic, as fans, coaches, players, and general managers often want to get a better understanding of how an individual is
impacting their team. It is one thing to watch the players to determine how good a player is, often refered to as the "eye test" but 
with so many players, it is hard to quantify this. Enter the plus-minus statistic. Plus-minus is generally calculated by adding all the points
scored by their team while they were playing and subtracting by points scored by the opposition while they were playing. The idea is that 
if a player has a generally positive impact on their team, they will have a highly positive plus-minus. The two main sports where this
statistic is used is basketball and ice hockey. 

While the plus-minus statistic is a great idea in theory, it does not come without it's weaknesses. The biggest drawback to this
statistic is its issues with independece. Take ice hockey for example. Many ice hockey players tend to play on "lines", meaning the same three 
forwards tend to play together and the same two defensemen play together. As a result, the performance of one player is highly dependent 
on the performance of their linemates. In "A Regression-based Adjusted Plus-Minus Statistic for NHL Players" \cite{Macdonald_2011}, Brian
MacDonald provides a perfect example of this with the Henrik and Daniel Sedin, Swedish twins who played together for the Vancouver Canucks.
Many critics of the plus-minus argue that plus-minus is more of a team statistic, since it is heavily influenced by team dynamics rather than
individual contributions. Furthermore, numerous confounding variables, such as the quality of the opponent and situational factors, make the 
calculation less reliable. As a result of these criticisms, alternatives to the plus-minus statistic have been developed, such as Corsi or 
Fenwick, which is calculated using shot attempts rather than goals.

The goal of this project will be to shed light on this controversial statistic and determine if plus-minus is truely an effective predictor
to individual contribution, specifically in ice hockey. 




% roadmap
The rest of the paper is organized as follows.
The data will be presented in Section~\ref{sec:data}.
The methods are described in Section~\ref{sec:meth}.
The results are reported in Section~\ref{sec:resu}.
A discussion concludes in Section~\ref{sec:disc}.


\section{Data}
\label{sec:data}

Use this section to describe the data that helps to answer your research
questions. Recall Einstein's equation
\begin{equation}
  \label{eq:mc2}
  E = m c^2,
\end{equation}
which states that the energy $E$ of a particle in its rest frame as the product
of mass ($m$) with the speed of light squared ($c^2$).
\lipsum{1}

\section{Methods}
\label{sec:meth}

Use this section to present the methodologies that will generate results by
analyzing the data. Suppose that the radius of a circle is $r$. Then its area is
\begin{equation}
  \label{eq:area}
  \pi r^2.
\end{equation}

Equation~\eqref{eq:area} is interesting. \lipsum[1-4]

Sometimes I don't want an equation to be numbered such as this one:
\[
  f(x) = \frac{1}{\sqrt{2\pi}} \exp\left( - \frac{x^2}{2} \right),
\]
which is the density of a standard normal variable.



\section{Results}
\label{sec:resu}

Table~\ref{tab:rv} summarizes some example draws from some distributions.
\lipsum[1-4]

\begin{table}[tbp]
  \caption{This is my first table.}
  \label{tab:rv}
\centering
\begin{tabular}{rrr}
  \toprule
normal & poisson & gamma \\ 
  \midrule
-0.110 & 4 & 2.401 \\ 
  0.116 & 4 & 3.529 \\ 
  -0.828 & 9 & 2.112 \\ 
  -0.066 & 6 & 11.104 \\ 
  0.219 & 3 & 4.815 \\ 
  0.303 & 5 & 2.188 \\ 
  0.544 & 0 & 8.050 \\ 
  -2.617 & 8 & 3.646 \\ 
  0.747 & 1 & 5.178 \\ 
  -1.103 & 4 & 3.043 \\ 
   \bottomrule
\end{tabular}
\end{table}


\section{Discussion}
\label{sec:disc}

What are the main contributions again?

What are the limitations of this study?

What are worth pursuing further in the future?

\lipsum[1-2]
Watch for prevalence of diabetes \citep{wild2004global}.

\bibliography{refs}
\bibliographystyle{mcap}

\end{document}