\documentclass[12pt]{article}

%% preamble: Keep it clean; only include those you need
\usepackage{amsmath}
\usepackage[margin = 1in]{geometry}
\usepackage{graphicx}
\usepackage{booktabs}
\usepackage{natbib}

%% for double spacing
\usepackage{setspace}

% for space filling
\usepackage{lipsum}
% highlighting hyper links
\usepackage[colorlinks=true, citecolor=blue]{hyperref}

% Get line numbers for ease of referencing for the reviewers
\usepackage[]{lineno}
\linenumbers*[1]
% patches to make lineno work better with amsmath
\newcommand*\patchAmsMathEnvironmentForLineno[1]{%
        \expandafter\let\csname old#1\expandafter\endcsname\csname 
        #1\endcsname
        \expandafter\let\csname oldend#1\expandafter\endcsname\csname 
        end#1\endcsname
        \renewenvironment{#1}%
        {\linenomath\csname old#1\endcsname}%
        {\csname oldend#1\endcsname\endlinenomath}}%
\newcommand*\patchBothAmsMathEnvironmentsForLineno[1]{%
        \patchAmsMathEnvironmentForLineno{#1}%
        \patchAmsMathEnvironmentForLineno{#1*}}%
\AtBeginDocument{%
        \patchBothAmsMathEnvironmentsForLineno{equation}%
        \patchBothAmsMathEnvironmentsForLineno{align}%
        \patchBothAmsMathEnvironmentsForLineno{flalign}%
        \patchBothAmsMathEnvironmentsForLineno{alignat}%
        \patchBothAmsMathEnvironmentsForLineno{gather}%
        \patchBothAmsMathEnvironmentsForLineno{multline}%
}



%% meta data

\title{Hockey's Most Controversial Statistic: An Analysis of the Effectiveness of the Plus-Minus Statistic}
\author{Melanie Desroches\\
  Department of Statistics\\
  University of Connecticut
}

\begin{document}
\maketitle

\begin{abstract}
Here is the abstract.  
\end{abstract}

\doublespacing

\section{Introduction}
\label{sec:intro}

Use this section to answer three questions:
Why is the topic important/interesting?
What has been done on this topic in the literature?
What is your contribution?

Within the realm of sports, many different statistics are used to determine how good a player is. This can include points,
assists, time played, etc. But is there just one valaue that can be used to determine the effectiveness of a player? This is an 
important topic, as fans, coaches, players, and general managers often want to get a better understanding of how an individual is
impacting their team. It is one thing to watch the players to determine how good a player is, often refered to as the "eye test" but 
with so many players, it is hard to quantify this. Enter the plus-minus statistic. Plus-minus is generally calculated by adding all the points
scored by their team while they were playing and subtracting by points scored by the opposition while they were playing. The idea is that 
if a player has a generally positive impact on their team, they will have a highly positive plus-minus. The two main sports where this
statistic is used is basketball and ice hockey. 

While the plus-minus statistic is a great idea in theory, it does not come without it's weaknesses. The biggest drawback to this
statistic is its issues with independece. Take ice hockey for example. Many ice hockey players tend to play on "lines", meaning the same three 
forwards tend to play together and the same two defensemen usually play together. As a result, the performance of one player is highly dependent 
on the performance of their linemates. In "A Regression-based Adjusted Plus-Minus Statistic for NHL Players" \cite{Macdonald_2011}, Brian
MacDonald provides a perfect example of this with the Henrik and Daniel Sedin, Swedish twins who played together for the Vancouver Canucks. 
 \cite[Daniel spent 92\% of his playing time with Henrik, the highest percentage of any other player combination where both players have
played over 700 minutes. Because of this high colinearity between the twins, it is difficult to separate the individual effect that each 
player has on the net goals scored on the ice.]{Macdonald_2012} Many critics of the plus-minus argue that plus-minus is more of a team 
statistic, since it is heavily influenced by team dynamics rather than individual contributions. Furthermore, numerous confounding variables, 
such as the quality of the opponent and situational factors, make the calculation less reliable. As a result of these criticisms, alternatives 
to the plus-minus statistic have been developed, such as Corsi or Fenwick, which is calculated using shot attempts rather than goals.

The goal of this project will be to shed light on this controversial statistic and determine if plus-minus is truely an effective predictor
to individual contribution, specifically in the National Hockey League (hereforth refered to as the NHL). Is the plus-minus a good evaluator 
of offensive and defensive output? Is the plus-minus more reflective of how well a player is doing than how well a player is preforming? 
Are there better alternatives to evaluating individual player contribution? These are the questions that will be answered from this analysis.


% roadmap
The rest of the paper is organized as follows.
The data will be presented in Section~\ref{sec:data}.
The methods are described in Section~\ref{sec:meth}.
The results are reported in Section~\ref{sec:resu}.
A discussion concludes in Section~\ref{sec:disc}.


\section{Data}
\label{sec:data}

Use this section to describe the data that helps to answer your research
questions.

The data used to perform this analysis was collected from Natural Stat Trick and the NHL website. The observations in the data
is from all players that played in the National Hockey League from the 2021-2022, 2022-2023, and 2023-2024 seasons. Only players
that have played in over 25 games were used in the analysis. This is ensure that the players being evaluated were all of similar 
calliber and were regular NHL players. The columns of the dataset will be combined from the two data sources. The following is a 
description of each of the columns in the dataset obtained from the Natural Stat Trick website:

Player - Player name.

Team - Team or teams that the player has played for. Not displayed when filtering for specific teams.

Position - Position or positions that the player has been listed as playing by the NHL.

GP - Games played.

TOI - Total amount of time played.

Corsi - Any shot attempt (goals, shots on net, misses and blocks) outside of the shootout. Referred to as SAT by the NHL.

CF - Count of Corsi for that player's team while that player is on the ice.

CA - Count of Corsi against that player's team while that player is on the ice.

CF\% - Percentage of total Corsi while that player is on the ice that are for that player's team. CF*100/(CF+CA)

Fenwick - any unblocked shot attempt (goals, shots on net and misses) outside of the shootout. Referred to as USAT by the NHL.

FF - Count of Fenwick for that player's team while that player is on the ice.

FA - Count of Fenwick against that player's team while that player is on the ice.

FF\% - Percentage of total Fenwick while that player is on the ice that are for that player's team. FF*100/(FF+FA)

Shots - any shot attempt on net (goals and shots on net) outside of the shootout.

SF - Count of Shots for that player's team while that player is on the ice.

SA - Count of Shots against that player's team while that player is on the ice.

SF\% - Percentage of total Shots while that player is on the ice that are for that player's team. SF*100/(SF+SA)

Goals - any goal, outside of the shootout.

GF - Count of Goals for that player's team while that player is on the ice.

GA - Count of Goals against that player's team while that player is on the ice.

GF\% - Percentage of total Goals while that player is on the ice that are for that player's team. GF*100/(GF+GA)

Scoring Chances - a scoring chance, as originally defined by War-on-Ice

SCF - Count of Scoring Chances for that player's team while that player is on the ice.

SCA - Count of Scoring Chances against that player's team while that player is on the ice.

SCF\% - Percentage of total Scoring Chances while that player is on the ice that are for that player's team. SCF*100/(SCF+SCA)

High Danger Scoring Chances - a scoring chance with a score of 3 or higher.

HDCF - Count of High Danger Scoring Chances for that player's team while that player is on the ice.

HDCA - Count of High Danger Scoring Chances against that player's team while that player is on the ice.

HDCF\% - Percentage of total High Danger Scoring Chances while that player is on the ice that are for that player's team. HDCF*100/(HDCF+HDCA)

High Danger Goals - goals generated from High Danger Scoring Chances

HDGF - Count of Goals off of High Danger Scoring Chances for that player's team while that player is on the ice.

HDGA - Count of Goals off of High Danger Scoring Chances against that player's team while that player is on the ice.

HDGF\% - Percentage of High Danger Goals while that player is on the ice that are for that player's team. HDGF*100/(HDGF+HDGA)

Medium Danger Scoring Chances - a scoring chance with a score of exactly 2.

MDCF - Count of Medium Danger Scoring Chances for that player's team while that player is on the ice.

MDCA - Count of Medium Danger Scoring Chances against that player's team while that player is on the ice.

MDCF\% - Percentage of total Medium Danger Scoring Chances while that player is on the ice that are for that player's team. MDCF*100/(MDCF+MDCA)

Medium Danger Goals - goals generated from Medium Danger Scoring Chances

MDGF - Count of Goals off of Medium Danger Scoring Chances for that player's team while that player is on the ice.

MDGA - Count of Goals off of Medium Danger Scoring Chances against that player's team while that player is on the ice.

MDGF\% - Percentage of Medium Danger Goals while that player is on the ice that are for that player's team. MDGF*100/(MDGF+MDGA)

Low Danger Scoring Chances - a scoring chance with a score of 1 or less. Does not include any attempts from the attacking team's neutral or defensive zone.

LDCF - Count of Low Danger Scoring Chances for that player's team while that player is on the ice.

LDCA - Count of Low Danger Scoring Chances against that player's team while that player is on the ice.

LDCF\% - Percentage of total Low Danger Scoring Chances while that player is on the ice that are for that player's team. LDCF*100/(LDCF+LDCA)

Low Danger Goals - goals generated from Low Danger Scoring Chances

LDGF - Count of Goals off of Low Danger Scoring Chances for that player's team while that player is on the ice.

LDGA - Count of Goals off of Low Danger Scoring Chances against that player's team while that player is on the ice.

LDGF\% - Percentage of Low Danger Goals while that player is on the ice that are for that player's team. LDGF*100/(LDGF+LDGA)

PDO

SH\% - Percentage of Shots for that player's team while that player is on the ice that were Goals. GF*100/SF

SV\% - Percentage of Shots against that player's team while that player is on the ice that were not Goals. GA*100/SA

PDO - Shooting percentage plus save percentage. (GF/SF)+(GA/SA)

Starts

Off. Zone Starts - Number of shifts for the player that started with an offensive zone faceoff.

Neu. Zone Starts - Number of shifts for the player that started with an neutral zone faceoff.

Def. Zone Starts - Number of shifts for the player that started with an defensive zone faceoff.

On The Fly Starts - Number of shifts for the player that started during play (without a faceoff).

Off. Zone Start \% - Percentage of starts for the player that were Offensive Zone Starts, excluding Neutral Zone and On The Fly Starts. Off. Zone Starts*100/(Off. Zone Starts+Def. Zone Starts)

Faceoffs

Off. Zone Faceoffs - Number of faceoffs in the offensive zone for which the player was on the ice.

Neu. Zone Faceoffs - Number of faceoffs in the neutral zone for which the player was on the ice.

Def. Zone Faceoffs - Number of faceoffs in the defensive zone for which the player was on the ice.

Off. Zone Faceoff \% - Percentage of faceoffs in the offensive zone for which the player was on the ice, excluding neutral zone faceoffs. Off. Zone Faceoffs*100/(Off. Zone Faceoffs+Def. Zone Faceoffs)

The goal of this dataset was to provide a broader scope of variables that evaluate a players performance. By introducing more
advanced and uncovential statistics, there are more factors to consider in evaluating a player's effectiveness on the ice.


\section{Methods}
\label{sec:meth}

Use this section to present the methodologies that will generate results by
analyzing the data. 

\subsection{Correlation Analysis}
The main issue with the data is multicollinearity, where independent variables are highly correlated. A correlation matrix was created 
with some of the more basic nhl statistics to highlight this issue, shown in Table \ref{tab:correlation_matrix}.

\begin{table}[h!]
  \centering
  \caption{Correlation Matrix of Variables}
  \begin{tabular}{lrrrrrrrrrrrrrrrrr}
\toprule
 & GP & G & A & P & +/- & P/GP & EVG & EVP & PPG & PPP & SHG & SHP & OTG & GWG & S & S% & TOI/GP \\
\midrule
GP & 1.000000 & 0.660694 & 0.742812 & 0.742844 & 0.212674 & 0.542885 & 0.694755 & 0.806373 & 0.475903 & 0.526041 & 0.410720 & 0.505980 & 0.442653 & 0.599482 & 0.826371 & 0.246746 & 0.559038 \\
G & 0.660694 & 1.000000 & 0.819358 & 0.933217 & 0.234048 & 0.881798 & 0.980843 & 0.926390 & 0.901738 & 0.844470 & 0.439135 & 0.417944 & 0.715858 & 0.922624 & 0.907769 & 0.594148 & 0.415454 \\
A & 0.742812 & 0.819358 & 1.000000 & 0.970627 & 0.335785 & 0.895312 & 0.800044 & 0.945635 & 0.748186 & 0.912095 & 0.350123 & 0.430347 & 0.708249 & 0.789043 & 0.892553 & 0.335478 & 0.680776 \\
P & 0.742844 & 0.933217 & 0.970627 & 1.000000 & 0.308682 & 0.931216 & 0.913072 & 0.981470 & 0.847371 & 0.926069 & 0.403738 & 0.445126 & 0.744332 & 0.881744 & 0.940538 & 0.459613 & 0.601041 \\
+/- & 0.212674 & 0.234048 & 0.335785 & 0.308682 & 1.000000 & 0.276643 & 0.235106 & 0.326182 & 0.186643 & 0.227844 & 0.180917 & 0.282005 & 0.158837 & 0.296291 & 0.246253 & 0.054788 & 0.238598 \\
P/GP & 0.542885 & 0.881798 & 0.895312 & 0.931216 & 0.276643 & 1.000000 & 0.851663 & 0.892999 & 0.832988 & 0.908954 & 0.328880 & 0.340711 & 0.725762 & 0.838393 & 0.835197 & 0.555692 & 0.572602 \\
EVG & 0.694755 & 0.980843 & 0.800044 & 0.913072 & 0.235106 & 0.851663 & 1.000000 & 0.932690 & 0.804992 & 0.778026 & 0.416574 & 0.401315 & 0.689839 & 0.901905 & 0.904258 & 0.598461 & 0.394952 \\
EVP & 0.806373 & 0.926390 & 0.945635 & 0.981470 & 0.326182 & 0.892999 & 0.932690 & 1.000000 & 0.783327 & 0.838735 & 0.408319 & 0.454309 & 0.709961 & 0.868988 & 0.949149 & 0.464550 & 0.591843 \\
PPG & 0.475903 & 0.901738 & 0.748186 & 0.847371 & 0.186643 & 0.832988 & 0.804992 & 0.783327 & 1.000000 & 0.884188 & 0.297616 & 0.285022 & 0.682860 & 0.840790 & 0.782334 & 0.495487 & 0.404067 \\
PPP & 0.526041 & 0.844470 & 0.912095 & 0.926069 & 0.227844 & 0.908954 & 0.778026 & 0.838735 & 0.884188 & 1.000000 & 0.280415 & 0.297894 & 0.736585 & 0.811334 & 0.816326 & 0.397766 & 0.550369 \\
SHG & 0.410720 & 0.439135 & 0.350123 & 0.403738 & 0.180917 & 0.328880 & 0.416574 & 0.408319 & 0.297616 & 0.280415 & 1.000000 & 0.875315 & 0.246458 & 0.390945 & 0.415902 & 0.301481 & 0.167273 \\
SHP & 0.505980 & 0.417944 & 0.430347 & 0.445126 & 0.282005 & 0.340711 & 0.401315 & 0.454309 & 0.285022 & 0.297894 & 0.875315 & 1.000000 & 0.243675 & 0.378260 & 0.453588 & 0.232071 & 0.293240 \\
OTG & 0.442653 & 0.715858 & 0.708249 & 0.744332 & 0.158837 & 0.725762 & 0.689839 & 0.709961 & 0.682860 & 0.736585 & 0.246458 & 0.243675 & 1.000000 & 0.757017 & 0.681360 & 0.327325 & 0.479141 \\
GWG & 0.599482 & 0.922624 & 0.789043 & 0.881744 & 0.296291 & 0.838393 & 0.901905 & 0.868988 & 0.840790 & 0.811334 & 0.390945 & 0.378260 & 0.757017 & 1.000000 & 0.838015 & 0.530423 & 0.422259 \\
S & 0.826371 & 0.907769 & 0.892553 & 0.940538 & 0.246253 & 0.835197 & 0.904258 & 0.949149 & 0.782334 & 0.816326 & 0.415902 & 0.453588 & 0.681360 & 0.838015 & 1.000000 & 0.374988 & 0.598176 \\
S% & 0.246746 & 0.594148 & 0.335478 & 0.459613 & 0.054788 & 0.555692 & 0.598461 & 0.464550 & 0.495487 & 0.397766 & 0.301481 & 0.232071 & 0.327325 & 0.530423 & 0.374988 & 1.000000 & -0.034103 \\
TOI/GP & 0.559038 & 0.415454 & 0.680776 & 0.601041 & 0.238598 & 0.572602 & 0.394952 & 0.591843 & 0.404067 & 0.550369 & 0.167273 & 0.293240 & 0.479141 & 0.422259 & 0.598176 & -0.034103 & 1.000000 \\
\bottomrule
\end{tabular}

  \label{tab:correlation_matrix}
\end{table}

There is a high correlation between different shot related statistics and point related statistics. For example, there is a 
0.826371 correlation between Goals and Shots. This makes since because in order to score a goal, the player needs to shoot first.
Correlation analysis can be used to identify the strength and direction of the relationship between various advanced metrics and 
the plus/minus statistic. Metrics with high positive correlations (e.g., GF\% and PDO) suggest a strong alignment with the plus/minus, 
indicating that they may reflect similar aspects of offensive or defensive performance. This was done by calculating the correlation 
coefficient between +/- and each metric (like CF\%, GF\%, SCF\%, etc.).The correlation values, ranging from -1 to 1, tell us how closely 
each metric aligns with the plus/minus. Higher positive correlations (e.g., with GF\% and PDO) indicate metrics that vary similarly to 
plus/minus, suggesting that they may capture overlapping aspects of offensive or defensive performance.

\subsection{Ridge Regression}

Due to the presence of correlation among variables, ridge regression was performed in order to identify how plus/minus can be employed 
to assess offensive and defensive contribution. It was also used to determine if other advanced metrics Corsi or Fenwick have the same, 
better, or worse predictive abilities compared to plus-minus. Ridge regression, a form of regularized linear regression, is beneficial 
in handling datasets where predictor variables are highly interrelated, as is the case with advanced hockey metrics. Ridge regression 
estimates the contribution of metrics like Corsi, Fenwick, and scoring chances while controlling for their interdependencies. This 
applies to hockey metrics like Corsi, Fenwick, scoring chances, etc. since these statistics often interact or overlap in measuring aspects 
of performance.

\subsection{Cross-Validation}
Coupled with ridge regression, cross validation was utilized to assess the predictive power of different groups of variables (offensive, 
defensive, possession-based) on the plus/minus statistic. Cross-validation splits the data into training and testing sets multiple times, 
computing a model's predictive accuracy each time. In this case, the data was split into five subsets. In each iteration, one subset was
kept as the test set and the other four were used to train the ridge regression model. The R-sqared value was reported for each iteration
to measure how well offensive, defensive, and possession-based metrics (like Corsi and Fenwick) predict plus/minus. This process ensures 
that the model’s performance is consistent across different data partitions.

\subsection{Mixed Model Effects}
In order to separate individual contributions to plus/minus from team-level effects, a mixed-effects model was used. Mixed-effects models 
allow for random effects, capturing variability at the team level, while fixed effects capture individual-level metrics. Individual metrics 
like CF\%, GF\%, and others were included that directly measure a player's on-ice performance. The model considered team-level averages 
(e.g., TeamMeanGF\%), accounting for how a player’s team context might influence their plus/minus.


\section{Results}
\label{sec:resu}

The results of taking the correlation coefficients of plus-minus versus every other column is as follows:


\begin{table}[tbp]
  \caption{Cross Validation Scores}
  \label{tab:rv}
\centering
\begin{tabular}{rrr}
  \toprule
Variable & Correlation Coefficient \\ 
  \midrule
  GP & 0.232144
  G & 0.237058
  A & 0.342538
  P & 0.314842
  +/- & 1.000000
  P/GP & 0.277666
  EVG & 0.239076
  EVP & 0.336372
  PPG & 0.187284
  PPP & 0.226803
  SHG & 0.177479
  SHP & 0.282570
  OTG & 0.161693
  GWG & 0.305998
  S & 0.255841
  S\% & 0.048282
  TOI/GP & 0.253234
  TOI & 0.291711
  CF & 0.369200
  CA & 0.210752
  CF\% & 0.533927
  FF & 0.373707
  FA & 0.213121
  FF\% & 0.553993
  SF & 0.376144
  SA & 0.211659
  SF\% & 0.568873
  GF & 0.423772
  GA & 0.083042
  GF\% & 0.705010
  xGF & 0.386643
  xGA & 0.195763
  xGF\% & 0.585988
  SCF & 0.387204
  SCA & 0.191645
  SCF\% & 0.585252
  HDCF & 0.399467
  HDCA & 0.195659
  HDCF\% & 0.551392
  HDGF & 0.442448
  HDGA & 0.092987
  HDGF\% & 0.614221
  On-Ice SH\% & 0.324061
  On-Ice SV\% & 0.419340
  PDO & 0.546471
  Off. Zone Starts & 0.309539
  Neu. Zone Starts & 0.294922
  Def. Zone Starts & 0.236777
  On The Fly Starts & 0.273141
  Off. Zone Start \% & 0.110750
  Off. Zone Faceoffs & 0.353819
  Neu. Zone Faceoffs & 0.314974
  Def. Zone Faceoffs & 0.262650
  Off. Zone Faceoff \% & 0.182999
  \bottomrule
\end{tabular}
\end{table}

Metrics such as GF\%, FF\%, SF\%, PDO, xGF\%, and HDGF\% show a relatively strong correlation with plus/minus (0.705010, 0.553993, 0.568873, 
0.546471, 0.585988, 0.614221 respectively), which aligns with expectations since many of these metrics also track goal-related contributions. 
Higher positive correlations indicate metrics that vary similarly to plus/minus, suggesting that they may capture overlapping aspects of
offensive or defensive performance.


Ridge regression was performed on three different sets of variables: combined metrics, offensive only, and defensive only. The combined
metrics consisted of Goals For Percentage(GF\%), Scoring Chances Percentage (SF\%), High-Danger Chances while that player is on the ice (HDSC\%),
combined shooting and save percentage (PDO), Percent of shots that were goals for (On-Ice SH\%), High-Danger Goals for while that player is 
on the ice (HDGC\%). All of these metrics are similar to plus-minus in that they combine the defensive and offensive factors together.
The offensive only metrics were CF, FF, SF, GF, SCF, HDCF, HDGF, GF\%, G, A, P/GP, CF\%, SCF\%, and On-Ice SH\%. The defensive only metrics were
CA, FA, SA, GA, SCF, HDCA, HDGA, On-Ice SV\%. The results can be seen in the table below.

Offensive/Defensive Combined: Averages around 0.53, indicating that these metrics explain a moderate proportion of the variance in plus/minus.
Offensive Only: Cross-validation scores are around 0.63, suggesting that offensive metrics alone are somewhat predictive of plus/minus, 
reflecting the offensive impact.
Defensive Only: With higher scores around 0.71, defensive metrics seem to have a greater influence on the plus/minus, which aligns with 
the role of plus/minus in capturing defensive contributions.
Corsi and Fenwick: Both show very high cross-validation scores, with Fenwick performing slightly better, indicating that these 
possession-based metrics are strong predictors.

\begin{table}[tbp]
  \caption{Cross Validation Scores}
  \label{tab:rv}
\centering
\begin{tabular}{rrr}
  \toprule
Both & Offensive & Defensive \\ 
  \midrule
  0.5696297 & 0.64692431 & 0.73265605 \\ 
  0.50125171 & 0.62667921 & 0.70862251 \\ 
  0.47617049 & 0.62263538 & 0.69467537 \\ 
  0.48944826 & 0.58484799 & 0.65315835 \\ 
  0.57586267 & 0.69145036 & 0.76239753 \\ 
  \bottomrule
\end{tabular}
\end{table}

Mixed Effects Model:


The significant coefficients for both GF\% and team averages, negative for TeamMeanCF\% and positive for TeamMeanGF\%, imply that 
team-level factors substantially impact individual plus/minus scores.
This aligns with the idea that plus/minus is both a team and individual statistic, as team-level scoring factors are significant predictors.

Plus-Minus vs Corsi vs Fenwick

Both Corsi and Fenwick were used as predictors in ridge regression with the variables GF\%, SCF\%, HDCF\%, PDO, On-Ice SH\%, On-Ice SV\%, G, A.
The results of the cross validation are shown in the table below

\begin{table}[tbp]
  \caption{Cross Validation Scores}
  \label{tab:rv}
\centering
\begin{tabular}{rrr}
  \toprule
Corsi & Fenwick \\ 
  \midrule
  0.93898296 & 0.95099819 \\ 
  0.90054897 & 0.93012992 \\ 
  0.93915158 & 0.9528202 \\ 
  0.92841222 & 0.95201592 \\ 
  0.9325304 & 0.95921829 \\ 
   \bottomrule
\end{tabular}
\end{table}

These values are notibly better that the cross validation scores for plus-minus, regardless of which set of predictors. This indicates
that both Corsi and Fenwick are significantly better at predicting player effectiveness than plus-minus.


\section{Discussion}
\label{sec:disc}

What are the main contributions again?

What are the limitations of this study?

What are worth pursuing further in the future?



\bibliography{refs}
\bibliographystyle{mcap}

\end{document}