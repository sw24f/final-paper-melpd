\documentclass[12pt]{article}

%% preamble: Keep it clean; only include those you need
\usepackage{amsmath}
\usepackage[margin = 1in]{geometry}
\usepackage{graphicx}
\usepackage{booktabs}
\usepackage{natbib}

%% for double spacing
\usepackage{setspace}

% for space filling
\usepackage{lipsum}
% highlighting hyper links
\usepackage[colorlinks=true, citecolor=blue]{hyperref}

% Get line numbers for ease of referencing for the reviewers
\usepackage[]{lineno}
\linenumbers*[1]
% patches to make lineno work better with amsmath
\newcommand*\patchAmsMathEnvironmentForLineno[1]{%
        \expandafter\let\csname old#1\expandafter\endcsname\csname 
        #1\endcsname
        \expandafter\let\csname oldend#1\expandafter\endcsname\csname 
        end#1\endcsname
        \renewenvironment{#1}%
        {\linenomath\csname old#1\endcsname}%
        {\csname oldend#1\endcsname\endlinenomath}}%
\newcommand*\patchBothAmsMathEnvironmentsForLineno[1]{%
        \patchAmsMathEnvironmentForLineno{#1}%
        \patchAmsMathEnvironmentForLineno{#1*}}%
\AtBeginDocument{%
        \patchBothAmsMathEnvironmentsForLineno{equation}%
        \patchBothAmsMathEnvironmentsForLineno{align}%
        \patchBothAmsMathEnvironmentsForLineno{flalign}%
        \patchBothAmsMathEnvironmentsForLineno{alignat}%
        \patchBothAmsMathEnvironmentsForLineno{gather}%
        \patchBothAmsMathEnvironmentsForLineno{multline}%
}



%% meta data

\title{Hockey's Most Controversial Statistic: An Analysis of the Effectiveness of the Plus-Minus Statistic}
\author{Melanie Desroches\\
  Department of Statistics\\
  University of Connecticut
}

\begin{document}
\maketitle

\begin{abstract}
Here is the abstract.  
\end{abstract}

\doublespacing

\section{Introduction}
\label{sec:intro}

Use this section to answer three questions:
Why is the topic important/interesting?
What has been done on this topic in the literature?
What is your contribution?

Within the realm of sports, many different statistics are used to determine how good a player is. This can include points,
assists, time played, etc. But is there just one valaue that can be used to determine the effectiveness of a player? This is an 
important topic, as fans, coaches, players, and general managers often want to get a better understanding of how an individual is
impacting their team. It is one thing to watch the players to determine how good a player is, often refered to as the "eye test" but 
with so many players, it is hard to quantify this. Enter the plus-minus statistic. Plus-minus is generally calculated by adding all the points
scored by their team while they were playing and subtracting by points scored by the opposition while they were playing. The idea is that 
if a player has a generally positive impact on their team, they will have a highly positive plus-minus. The two main sports where this
statistic is used is basketball and ice hockey. 

While the plus-minus statistic is a great idea in theory, it does not come without it's weaknesses. The biggest drawback to this
statistic is its issues with independece. Take ice hockey for example. Many ice hockey players tend to play on "lines", meaning the same three 
forwards tend to play together and the same two defensemen usually play together. As a result, the performance of one player is highly dependent 
on the performance of their linemates. In "A Regression-based Adjusted Plus-Minus Statistic for NHL Players" \cite{Macdonald_2011}, Brian
MacDonald provides a perfect example of this with the Henrik and Daniel Sedin, Swedish twins who played together for the Vancouver Canucks. 
 \cite[Daniel spent 92\% of his playing time with Henrik, the highest percentage of any other player combination where both players have
played over 700 minutes. Because of this high colinearity between the twins, it is difficult to separate the individual effect that each 
player has on the net goals scored on the ice.]{Macdonald_2012} Many critics of the plus-minus argue that plus-minus is more of a team 
statistic, since it is heavily influenced by team dynamics rather than individual contributions. Furthermore, numerous confounding variables, 
such as the quality of the opponent and situational factors, make the calculation less reliable. As a result of these criticisms, alternatives 
to the plus-minus statistic have been developed, such as Corsi or Fenwick, which is calculated using shot attempts rather than goals.

The goal of this project will be to shed light on this controversial statistic and determine if plus-minus is truely an effective predictor
to individual contribution, specifically in the National Hockey League (hereforth refered to as the NHL). Is the plus-minus a good evaluator 
of offensive and defensive output? Is the plus-minus more reflective of how well a player is doing than how well a player is preforming? 
Are there better alternatives to evaluating individual player contribution? These are the questions that will be answered from this analysis.




% roadmap
The rest of the paper is organized as follows.
The data will be presented in Section~\ref{sec:data}.
The methods are described in Section~\ref{sec:meth}.
The results are reported in Section~\ref{sec:resu}.
A discussion concludes in Section~\ref{sec:disc}.


\section{Data}
\label{sec:data}

Use this section to describe the data that helps to answer your research
questions.

The data used to perform this analysis was collected from Natural Stat Trick and the NHL website. The observations in the data
is from all players that played in the National Hockey League from the 2021-2022, 2022-2023, and 2023-2024 seasons. The columns
of the dataset will be combined from the two data sources. The following is a description of each of the columns in the dataset
obtained from the Natural Stat Trick website:

Player - Player name.

Team - Team or teams that the player has played for. Not displayed when filtering for specific teams.

Position - Position or positions that the player has been listed as playing by the NHL.

GP - Games played.

TOI - Total amount of time played.

Corsi - Any shot attempt (goals, shots on net, misses and blocks) outside of the shootout. Referred to as SAT by the NHL.

CF - Count of Corsi for that player's team while that player is on the ice.

CA - Count of Corsi against that player's team while that player is on the ice.

CF\% - Percentage of total Corsi while that player is on the ice that are for that player's team. CF*100/(CF+CA)

Fenwick - any unblocked shot attempt (goals, shots on net and misses) outside of the shootout. Referred to as USAT by the NHL.

FF - Count of Fenwick for that player's team while that player is on the ice.

FA - Count of Fenwick against that player's team while that player is on the ice.

FF\% - Percentage of total Fenwick while that player is on the ice that are for that player's team. FF*100/(FF+FA)

Shots - any shot attempt on net (goals and shots on net) outside of the shootout.

SF - Count of Shots for that player's team while that player is on the ice.

SA - Count of Shots against that player's team while that player is on the ice.

SF\% - Percentage of total Shots while that player is on the ice that are for that player's team. SF*100/(SF+SA)

Goals - any goal, outside of the shootout.

GF - Count of Goals for that player's team while that player is on the ice.

GA - Count of Goals against that player's team while that player is on the ice.

GF\% - Percentage of total Goals while that player is on the ice that are for that player's team. GF*100/(GF+GA)

Scoring Chances - a scoring chance, as originally defined by War-on-Ice

SCF - Count of Scoring Chances for that player's team while that player is on the ice.

SCA - Count of Scoring Chances against that player's team while that player is on the ice.

SCF\% - Percentage of total Scoring Chances while that player is on the ice that are for that player's team. SCF*100/(SCF+SCA)

High Danger Scoring Chances - a scoring chance with a score of 3 or higher.

HDCF - Count of High Danger Scoring Chances for that player's team while that player is on the ice.

HDCA - Count of High Danger Scoring Chances against that player's team while that player is on the ice.

HDCF\% - Percentage of total High Danger Scoring Chances while that player is on the ice that are for that player's team. HDCF*100/(HDCF+HDCA)

High Danger Goals - goals generated from High Danger Scoring Chances

HDGF - Count of Goals off of High Danger Scoring Chances for that player's team while that player is on the ice.

HDGA - Count of Goals off of High Danger Scoring Chances against that player's team while that player is on the ice.

HDGF\% - Percentage of High Danger Goals while that player is on the ice that are for that player's team. HDGF*100/(HDGF+HDGA)

Medium Danger Scoring Chances - a scoring chance with a score of exactly 2.

MDCF - Count of Medium Danger Scoring Chances for that player's team while that player is on the ice.

MDCA - Count of Medium Danger Scoring Chances against that player's team while that player is on the ice.

MDCF\% - Percentage of total Medium Danger Scoring Chances while that player is on the ice that are for that player's team. MDCF*100/(MDCF+MDCA)

Medium Danger Goals - goals generated from Medium Danger Scoring Chances

MDGF - Count of Goals off of Medium Danger Scoring Chances for that player's team while that player is on the ice.

MDGA - Count of Goals off of Medium Danger Scoring Chances against that player's team while that player is on the ice.

MDGF\% - Percentage of Medium Danger Goals while that player is on the ice that are for that player's team. MDGF*100/(MDGF+MDGA)

Low Danger Scoring Chances - a scoring chance with a score of 1 or less. Does not include any attempts from the attacking team's neutral or defensive zone.

LDCF - Count of Low Danger Scoring Chances for that player's team while that player is on the ice.

LDCA - Count of Low Danger Scoring Chances against that player's team while that player is on the ice.

LDCF\% - Percentage of total Low Danger Scoring Chances while that player is on the ice that are for that player's team. LDCF*100/(LDCF+LDCA)

Low Danger Goals - goals generated from Low Danger Scoring Chances

LDGF - Count of Goals off of Low Danger Scoring Chances for that player's team while that player is on the ice.

LDGA - Count of Goals off of Low Danger Scoring Chances against that player's team while that player is on the ice.

LDGF\% - Percentage of Low Danger Goals while that player is on the ice that are for that player's team. LDGF*100/(LDGF+LDGA)

PDO

SH\% - Percentage of Shots for that player's team while that player is on the ice that were Goals. GF*100/SF

SV\% - Percentage of Shots against that player's team while that player is on the ice that were not Goals. GA*100/SA

PDO - Shooting percentage plus save percentage. (GF/SF)+(GA/SA)

Starts

Off. Zone Starts - Number of shifts for the player that started with an offensive zone faceoff.

Neu. Zone Starts - Number of shifts for the player that started with an neutral zone faceoff.

Def. Zone Starts - Number of shifts for the player that started with an defensive zone faceoff.

On The Fly Starts - Number of shifts for the player that started during play (without a faceoff).

Off. Zone Start \% - Percentage of starts for the player that were Offensive Zone Starts, excluding Neutral Zone and On The Fly Starts. Off. Zone Starts*100/(Off. Zone Starts+Def. Zone Starts)

Faceoffs

Off. Zone Faceoffs - Number of faceoffs in the offensive zone for which the player was on the ice.

Neu. Zone Faceoffs - Number of faceoffs in the neutral zone for which the player was on the ice.

Def. Zone Faceoffs - Number of faceoffs in the defensive zone for which the player was on the ice.

Off. Zone Faceoff \% - Percentage of faceoffs in the offensive zone for which the player was on the ice, excluding neutral zone faceoffs. Off. Zone Faceoffs*100/(Off. Zone Faceoffs+Def. Zone Faceoffs)


\section{Methods}
\label{sec:meth}

Use this section to present the methodologies that will generate results by
analyzing the data. Suppose that the radius of a circle is $r$. Then its area is
\begin{equation}
  \label{eq:area}
  \pi r^2.
\end{equation}

Equation~\eqref{eq:area} is interesting. \lipsum[1-4]

Sometimes I don't want an equation to be numbered such as this one:
\[
  f(x) = \frac{1}{\sqrt{2\pi}} \exp\left( - \frac{x^2}{2} \right),
\]
which is the density of a standard normal variable.



\section{Results}
\label{sec:resu}

Table~\ref{tab:rv} summarizes some example draws from some distributions.
\lipsum[1-4]

\begin{table}[tbp]
  \caption{This is my first table.}
  \label{tab:rv}
\centering
\begin{tabular}{rrr}
  \toprule
normal & poisson & gamma \\ 
  \midrule
-0.110 & 4 & 2.401 \\ 
  0.116 & 4 & 3.529 \\ 
  -0.828 & 9 & 2.112 \\ 
  -0.066 & 6 & 11.104 \\ 
  0.219 & 3 & 4.815 \\ 
  0.303 & 5 & 2.188 \\ 
  0.544 & 0 & 8.050 \\ 
  -2.617 & 8 & 3.646 \\ 
  0.747 & 1 & 5.178 \\ 
  -1.103 & 4 & 3.043 \\ 
   \bottomrule
\end{tabular}
\end{table}


\section{Discussion}
\label{sec:disc}

What are the main contributions again?

What are the limitations of this study?

What are worth pursuing further in the future?

\lipsum[1-2]
Watch for prevalence of diabetes \citep{wild2004global}.

\bibliography{refs}
\bibliographystyle{mcap}

\end{document}