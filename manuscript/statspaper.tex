\documentclass[12pt]{article}

%% preamble: Keep it clean; only include those you need
\usepackage{amsmath}
\usepackage[margin = 1in]{geometry}
\usepackage{graphicx}
\usepackage{booktabs}
\usepackage{natbib}

%% for double spacing
\usepackage{setspace}

% for space filling
\usepackage{lipsum}
% highlighting hyper links
\usepackage[colorlinks=true, citecolor=blue]{hyperref}

% Get line numbers for ease of referencing for the reviewers
\usepackage[]{lineno}
\linenumbers*[1]
% patches to make lineno work better with amsmath
\newcommand*\patchAmsMathEnvironmentForLineno[1]{%
        \expandafter\let\csname old#1\expandafter\endcsname\csname 
        #1\endcsname
        \expandafter\let\csname oldend#1\expandafter\endcsname\csname 
        end#1\endcsname
        \renewenvironment{#1}%
        {\linenomath\csname old#1\endcsname}%
        {\csname oldend#1\endcsname\endlinenomath}}%
\newcommand*\patchBothAmsMathEnvironmentsForLineno[1]{%
        \patchAmsMathEnvironmentForLineno{#1}%
        \patchAmsMathEnvironmentForLineno{#1*}}%
\AtBeginDocument{%
        \patchBothAmsMathEnvironmentsForLineno{equation}%
        \patchBothAmsMathEnvironmentsForLineno{align}%
        \patchBothAmsMathEnvironmentsForLineno{flalign}%
        \patchBothAmsMathEnvironmentsForLineno{alignat}%
        \patchBothAmsMathEnvironmentsForLineno{gather}%
        \patchBothAmsMathEnvironmentsForLineno{multline}%
}



%% meta data

\title{Hockey's Most Controversial Statistic: An Analysis of the Effectiveness of the Plus-Minus Statistic}
\author{Melanie Desroches\\
  Department of Statistics\\
  University of Connecticut
}

\begin{document}
\maketitle

\begin{abstract}

The goal of this paper is to evaluate the plus minus statistic as a metric for determining the effectiveness of a player.
This involves analysing the relationship between a players plus-minus rating and other offensive and defensive ratings,
the effect of a players team performance on their plus-minus, and comparing plus-minus to other similar statistics. Based
on the analysis performed, the plus minus does have some value as a predictor of offensive and defensive metrics. However, other
advanced metrics like Corsi and Fenwick perform much better. The plus-minus is also heavily influenced by a player's team, making it 
less valuable as an indivual evaluator of player effectiveness.

\end{abstract}

\doublespacing

\section{Introduction}
\label{sec:intro}


Within the realm of sports, many different statistics are used to determine how good a player is. This can include points,
assists, time played, etc. But is there just one value that can be used to determine the effectiveness of a player? This is an 
important topic, as fans, coaches, players, and general managers often want to get a better understanding of how an individual is
impacting their team. It is one thing to watch the players to determine how good a player is, often referred to as the "eye test", but 
with so many players, it is hard to quantify this. Enter the plus-minus statistic. Plus-minus is generally calculated by adding all the points
scored by their team while they were playing and subtracting by points scored by the opposition while they were playing. The idea is that 
if a player has a generally positive impact on their team, they will have a highly positive plus-minus. In this project, plus-minus will
be evaluated in the context of ice hockey in the National Hockey League (henceforth refered to as the NHL).

The use of the plus-minus originated in ice hockey by the NHL team Montréal Canadiens \cite{plus_minus_rating}. The league
started to keep track of plus-minus in 1956. The NHL defines plus-minus as "a team's goal differential while a particular player 
is on the ice, excluding power play goals for and against but including empty net situations. All the skaters on the ice receive a plus 
or minus when an even-strength goal or shorthanded goal is scored depending on which team scored" \cite{nhl_website}. This means
that when a goal is scored, as long as there are the same number of players from each team on the ice, the team that scored gets plus one
and the other team gets a minus one.

While the plus-minus statistic is a great idea in theory, it does not come without its weaknesses. The biggest drawback to this
statistic is its issues with independence. Many ice hockey players tend to play on "lines", meaning the same three 
forwards tend to play together and the same two defensemen usually play together. As a result, the performance of one player is highly dependent 
on the performance of their linemates. In "A Regression-based Adjusted Plus-Minus Statistic for NHL Players" \cite{Macdonald_2011}, Brian
MacDonald provides a perfect example of this with the Henrik and Daniel Sedin, Swedish twins who played together for the Vancouver Canucks. 
"Daniel spent 92\% of his playing time with Henrik, the highest percentage of any other player combination where both players have
played over 700 minutes. Because of this high collinearity between the twins, it is difficult to separate the individual effect that each 
player has on the net goals scored on the ice."\cite{Macdonald_2012} Many critics of the plus-minus argue that plus-minus is more of a team 
statistic, since it is heavily influenced by team dynamics rather than individual contributions. Furthermore, numerous confounding variables, 
such as the quality of the opponent and situational factors, make the calculation less reliable. 

As a result of these criticisms, alternatives to the plus-minus statistic have been developed. Many of these alternatives are either different 
metrics or adjusted plus-minus\cite{plus_minus_rating}. Two popular examples are Corsi or Fenwick, which are calculated using shot attempts 
rather than goals. Another popular metric is expected goals, which is based on the probability of a goal being scored depending on the shot.
Brian MacDonald introduced an adjusted plus-minus statistic that is based on weighted least squares regression\cite{Macdonald_2012}. However,
many of these alternatives are more complicated to calculate, which can be confusing to the average fan. Due to the fact that plus-minus
is easy to interpret and calculate, it has remained in use despite its controversial background.

The goal of this project will be to shed light on this controversial statistic and determine if plus-minus is truly an effective predictor
to individual contribution, specifically in the National Hockey League (henceforth refered to as the NHL). Is the plus-minus a good evaluator 
of offensive and defensive output? Is the plus-minus more reflective of how well a player is doing than how well a player is performing? 
Are there better alternatives to evaluating individual player contribution? Is the plus-minus an effective predictor to individual contribution? 
These are the questions that will be answered from this analysis.


% roadmap
The rest of the paper is organized as follows.
The data will be presented in Section~\ref{sec:data}.
The methods are described in Section~\ref{sec:meth}.
The results are reported in Section~\ref{sec:resu}.
A discussion concludes in Section~\ref{sec:disc}.


\section{Data}
\label{sec:data}

The data used to perform this analysis was collected from Natural Stat Trick and the NHL website. The observations in the data
is from all players that played in the National Hockey League from the 2021-2022, 2022-2023, and 2023-2024 seasons. Only players
that have played in over 25 games were used in the analysis. This is to ensure that the players being evaluated were all of similar 
calliber and were regular NHL players. The columns of the dataset will be combined from the two data sources. The following is a 
description of each of the columns in the dataset obtained from the website \cite{natural_stat_trick}:

Player - Player name.

Team - Team or teams that the player has played for. Not displayed when filtering for specific teams.

Position - Position or positions that the player has been listed as playing by the NHL.

GP - Games played.

TOI - Total amount of time played.

Corsi - Any shot attempt (goals, shots on net, misses and blocks) outside of the shootout. Referred to as SAT by the NHL.

CF - Count of Corsi for that player's team while that player is on the ice.

CA - Count of Corsi against that player's team while that player is on the ice.

CF\% - Percentage of total Corsi while that player is on the ice that are for that player's team. CF*100/(CF+CA)

Fenwick - any unblocked shot attempt (goals, shots on net and misses) outside of the shootout. Referred to as USAT by the NHL.

FF - Count of Fenwick for that player's team while that player is on the ice.

FA - Count of Fenwick against that player's team while that player is on the ice.

FF\% - Percentage of total Fenwick while that player is on the ice that are for that player's team. FF*100/(FF+FA)

Shots - any shot attempt on net (goals and shots on net) outside of the shootout.

SF - Count of Shots for that player's team while that player is on the ice.

SA - Count of Shots against that player's team while that player is on the ice.

SF\% - Percentage of total Shots while that player is on the ice that are for that player's team. SF*100/(SF+SA)

Goals - any goal, outside of the shootout.

GF - Count of Goals for that player's team while that player is on the ice.

GA - Count of Goals against that player's team while that player is on the ice.

GF\% - Percentage of total Goals while that player is on the ice that are for that player's team. GF*100/(GF+GA)

Scoring Chances - a scoring chance, as originally defined by War-on-Ice

SCF - Count of Scoring Chances for that player's team while that player is on the ice.

SCA - Count of Scoring Chances against that player's team while that player is on the ice.

SCF\% - Percentage of total Scoring Chances while that player is on the ice that are for that player's team. SCF*100/(SCF+SCA)

High Danger Scoring Chances - a scoring chance with a score of 3 or higher.

HDCF - Count of High Danger Scoring Chances for that player's team while that player is on the ice.

HDCA - Count of High Danger Scoring Chances against that player's team while that player is on the ice.

HDCF\% - Percentage of total High Danger Scoring Chances while that player is on the ice that are for that player's team. HDCF*100/(HDCF+HDCA)

High Danger Goals - goals generated from High Danger Scoring Chances

HDGF - Count of Goals off of High Danger Scoring Chances for that player's team while that player is on the ice.

HDGA - Count of Goals off of High Danger Scoring Chances against that player's team while that player is on the ice.

HDGF\% - Percentage of High Danger Goals while that player is on the ice that are for that player's team. HDGF*100/(HDGF+HDGA)

SH\% - Percentage of Shots for that player's team while that player is on the ice that were Goals. GF*100/SF

SV\% - Percentage of Shots against that player's team while that player is on the ice that were not Goals. GA*100/SA

PDO - Shooting percentage plus save percentage. (GF/SF)+(GA/SA)

SH\% - Percentage of Shots for that player's team while that player is on the ice that were Goals. GF*100/SF

SV\% - Percentage of Shots against that player's team while that player is on the ice that were not Goals. GA*100/SA

Goals - Goals scored by the player, outside of the shootout.

Assists - Any assist by the player.

First Assists - Primary assists by the player.

Second Assists - Secondary assists by the player.

Total Points - Goals scored and assists by the player, outside of the shootout.

IPP - Individual Point Percentage, the percentage of goals for that player's team while that player is on the ice that the 
player earned a point on. 

Shots - Any shot attempt on net (goals and shots on net) by the player, outside of the shootout.

SH\% - Percentage of Shots by the player that were Goals. Goals*100/Shots

iCF - Any shot attempt (goals, shots on net, misses and blocks) by the player, outside of the shootout.

iFF - Any unblocked shot attempt (goals, shots on net and misses) by the player, outside of the shootout.

iSCF - Any scoring chance by the player, outside of the shootout.

iHDCF - Any high danger scoring chance by the player, outside of the shootout.

Rush Attempts - Any rush shot attempt (goals, shots on net, misses and blocks) by the player, outside of the shootout.

Rebounds Created - Any shot attempt (shots on net, misses and blocks) that results in a rebound shot attempt.

Penalties Drawn - Number of penalties committed against the player.

Giveaways - Number of unforced turnovers made by the player.

Takeaways - Number of times the player takes the puck away from the opposition.

Hits - Number of hits made by the player.

Hits Taken - Number of hits taken by the player.

Shots Blocked - Number of opposition shot attempts blocked by the player.


The goal of this dataset was to provide a broader scope of variables that evaluate a players performance. By introducing more
advanced and unconventional statistics, there are more factors to consider in evaluating a player's effectiveness on the ice.


\section{Methods}
\label{sec:meth}

\subsection{Correlation Analysis}
The main issue with the data is multicollinearity, where independent variables are highly correlated. A correlation matrix was created 
with some of the more basic nhl statistics to highlight this issue, shown in Table \ref{tab:correlation_matrix2}.

\begin{table}[h!]
  \centering
  \scriptsize
  \caption{Correlation Matrix of Variables}
  \begin{tabular}{lrrrrrrrrrrrrrrrrr}
\toprule
 & GP & G & A & P & +/- & P/GP & EVG & EVP & PPG & PPP & SHG & SHP & OTG & GWG & S & S\% & TOI \\
\midrule
GP & 1.00 & 0.66 & 0.74 & 0.74 & 0.21 & 0.54 & 0.69 & 0.81 & 0.48 & 0.53 & 0.41 & 0.51 & 0.44 & 0.60 & 0.83 & 0.25 & 0.56 \\
G & 0.66 & 1.00 & 0.82 & 0.93 & 0.23 & 0.88 & 0.98 & 0.93 & 0.90 & 0.84 & 0.44 & 0.42 & 0.72 & 0.92 & 0.91 & 0.59 & 0.42 \\
A & 0.74 & 0.82 & 1.00 & 0.97 & 0.34 & 0.90 & 0.80 & 0.95 & 0.75 & 0.91 & 0.35 & 0.43 & 0.71 & 0.79 & 0.89 & 0.34 & 0.68 \\
P & 0.74 & 0.93 & 0.97 & 1.00 & 0.31 & 0.93 & 0.91 & 0.98 & 0.85 & 0.93 & 0.40 & 0.45 & 0.74 & 0.88 & 0.94 & 0.46 & 0.60 \\
+/- & 0.21 & 0.23 & 0.34 & 0.31 & 1.00 & 0.28 & 0.24 & 0.33 & 0.19 & 0.23 & 0.18 & 0.28 & 0.16 & 0.30 & 0.25 & 0.05 & 0.24 \\
P/GP & 0.54 & 0.88 & 0.90 & 0.93 & 0.28 & 1.00 & 0.85 & 0.89 & 0.83 & 0.91 & 0.33 & 0.34 & 0.73 & 0.84 & 0.84 & 0.56 & 0.57 \\
EVG & 0.69 & 0.98 & 0.80 & 0.91 & 0.24 & 0.85 & 1.00 & 0.93 & 0.80 & 0.78 & 0.42 & 0.40 & 0.69 & 0.90 & 0.90 & 0.60 & 0.39 \\
EVP & 0.81 & 0.93 & 0.95 & 0.98 & 0.33 & 0.89 & 0.93 & 1.00 & 0.78 & 0.84 & 0.41 & 0.45 & 0.71 & 0.87 & 0.95 & 0.46 & 0.59 \\
PPG & 0.48 & 0.90 & 0.75 & 0.85 & 0.19 & 0.83 & 0.80 & 0.78 & 1.00 & 0.88 & 0.30 & 0.29 & 0.68 & 0.84 & 0.78 & 0.50 & 0.40 \\
PPP & 0.53 & 0.84 & 0.91 & 0.93 & 0.23 & 0.91 & 0.78 & 0.84 & 0.88 & 1.00 & 0.28 & 0.30 & 0.74 & 0.81 & 0.82 & 0.40 & 0.55 \\
SHG & 0.41 & 0.44 & 0.35 & 0.40 & 0.18 & 0.33 & 0.42 & 0.41 & 0.30 & 0.28 & 1.00 & 0.88 & 0.25 & 0.39 & 0.42 & 0.30 & 0.17 \\
SHP & 0.51 & 0.42 & 0.43 & 0.45 & 0.28 & 0.34 & 0.40 & 0.45 & 0.29 & 0.30 & 0.88 & 1.00 & 0.24 & 0.38 & 0.45 & 0.23 & 0.29 \\
OTG & 0.44 & 0.72 & 0.71 & 0.74 & 0.16 & 0.73 & 0.69 & 0.71 & 0.68 & 0.74 & 0.25 & 0.24 & 1.00 & 0.76 & 0.68 & 0.33 & 0.48 \\
GWG & 0.60 & 0.92 & 0.79 & 0.88 & 0.30 & 0.84 & 0.90 & 0.87 & 0.84 & 0.81 & 0.39 & 0.38 & 0.76 & 1.00 & 0.84 & 0.53 & 0.42 \\
S & 0.83 & 0.91 & 0.89 & 0.94 & 0.25 & 0.84 & 0.90 & 0.95 & 0.78 & 0.82 & 0.42 & 0.45 & 0.68 & 0.84 & 1.00 & 0.37 & 0.60 \\
S\% & 0.25 & 0.59 & 0.34 & 0.46 & 0.05 & 0.56 & 0.60 & 0.46 & 0.50 & 0.40 & 0.30 & 0.23 & 0.33 & 0.53 & 0.37 & 1.00 & -0.03 \\
TOI & 0.56 & 0.42 & 0.68 & 0.60 & 0.24 & 0.57 & 0.39 & 0.59 & 0.40 & 0.55 & 0.17 & 0.29 & 0.48 & 0.42 & 0.60 & -0.03 & 1.00 \\
\bottomrule
\end{tabular}

  \label{tab:correlation_matrix2}
\end{table}


There is a high correlation between different shot related statistics and point related statistics. For example, there is a 
0.826371 correlation between Goals and Shots. This makes sense because in order to score a goal, the player needs to shoot first.
Correlation analysis can be used to identify the strength and direction of the relationship between various advanced metrics and 
the plus/minus statistic. Metrics with high positive correlations (e.g., GF\% and PDO) suggest a strong alignment with the plus/minus, 
indicating that they may reflect similar aspects of offensive or defensive performance. This was done by calculating the correlation 
coefficient between +/- and each metric (like CF\%, GF\%, SCF\%, etc.).The correlation values, ranging from -1 to 1, tell us how closely 
each metric aligns with the plus/minus. Higher positive correlations (e.g., with GF\% and PDO) indicate metrics that vary similarly to 
plus/minus, suggesting that they may capture overlapping aspects of offensive or defensive performance.

\subsection{Ridge Regression}

Ridge regression is used in this project to evaluate the relationship between various performance metrics and plus-minus, while addressing 
multicollinearity among predictor variables. This regularization technique is particularly valuable in this context because many hockey 
performance metrics, such as Scoring Chances For percentage (SCF\%) and Goals For percentage (GF\%), are often highly correlated.
Due to the presence of correlation among variables, ridge regression was performed in order to identify how plus/minus can be employed 
to assess offensive and defensive contribution. It was also used to determine if other advanced metrics Corsi or Fenwick have the same, 
better, or worse predictive abilities compared to plus-minus. Ridge regression, a form of regularized linear regression, is beneficial 
in handling datasets where predictor variables are highly interrelated, as is the case with advanced hockey metrics. Ridge regression 
estimates the contribution of metrics like Corsi, Fenwick, and scoring chances while controlling for their interdependencies. This 
applies to hockey metrics like Corsi, Fenwick, scoring chances, etc. since these statistics often interact or overlap in measuring aspects 
of performance. Ridge regression is appropriate because the data satisfies the model assumptions of independence, homoscedasticity, 
and linearity. It is ideal for this project because it effectively manages multicollinearity, retains all variables for interpretability, 
and improves predictive accuracy. Other regularization techniques like Lasso or Elastic Net are better suited for sparse models or when 
variable selection is a priority, which is not the project's focus. 

\subsection{Cross-Validation}

Coupled with ridge regression, cross validation was utilized to assess the predictive power of different groups of variables (offensive, 
defensive, possession-based) on the plus/minus statistic. Cross-validation splits the data into training and testing sets multiple times, 
computing a model's predictive accuracy each time. In this case, the data was split into five subsets. In each iteration, one subset was
kept as the test set and the other four were used to train the ridge regression model. The R-squared value was reported for each iteration
to measure how well offensive, defensive, and possession-based metrics (like Corsi and Fenwick) predict plus/minus. This process ensures 
that the model’s performance is consistent across different data partitions.

\subsection{Mixed Model Effects}

In order to separate individual contributions to plus/minus from team-level effects, a mixed-effects model was used. 
The mixed-effects model combines:
\begin{itemize}
    \item \textbf{Fixed Effects}, which capture the influence of player-specific variables that directly relate to individual performance, 
    such as Corsi For Percentage (CF\%), Goals For Percentage (GF\%), and other advanced metrics.
    \item \textbf{Random Effects}, which account for variability at the team level, recognizing that a player’s plus/minus statistic can be 
    influenced by the overall performance and style of their team. Including team-level random effects helps control for unobserved team 
    factors that may affect each player similarly.
\end{itemize}

To apply this model, individual metrics that describe on-ice performance (e.g., CF\%, GF\%) were used as fixed effects to estimate each player’s 
contribution to plus/minus. At the same time, team-level averages (e.g., TeamMeanCF\%, TeamMeanGF\%) were included as random effects. This 
approach distinguishes how much of a player's plus/minus statistic is attributable to their own performance versus the performance of their team.
By using this mixed-effects approach, we can assess to what extent the plus/minus statistic reflects individual skill as opposed to team 
strength, thus helping clarify if plus/minus can serve as a reliable individual performance measure. The model was tested to ensure that the
assumptions of linearity, independence, normality, homoscedasticity, and random sampling were not violated.


\section{Results}
\label{sec:resu}

The results of the correlation analysis and statistical modeling of plus-minus reveal several insights into how this statistic reflects 
both individual and team-level contributions.

\subsection{Correlation Analysis}

The initial analysis involved calculating the correlation coefficients between plus-minus and various performance metrics. The correlation 
coefficients are displayed in Table \ref{tab:rv1}.


\begin{table}[tbp]
  \caption{Correlation Coefficients with Plus-Minus}
  \label{tab:rv1}
\centering
\small
\begin{tabular}{rr}
  \toprule
Variable & Correlation Coefficient \\ 
  \midrule
  GP & 0.232 \\ 
  G & 0.237 \\ 
  A & 0.343 \\ 
  P & 0.315 \\ 
  +/- & 1.000 \\ 
  P/GP & 0.278 \\ 
  EVG & 0.239 \\ 
  EVP & 0.336 \\ 
  PPG & 0.187 \\ 
  PPP & 0.227 \\ 
  SHG & 0.177 \\ 
  SHP & 0.283 \\ 
  OTG & 0.162 \\ 
  GWG & 0.306 \\ 
  S & 0.256 \\ 
  S\% & 0.048 \\ 
  TOI & 0.292 \\ 
  CF & 0.369 \\ 
  CA & 0.211 \\ 
  CF\% & 0.534 \\ 
  FF & 0.374 \\ 
  FA & 0.213 \\ 
  FF\% & 0.554 \\ 
  SF & 0.376 \\ 
  SA & 0.212 \\ 
  SF\% & 0.569 \\ 
  GF & 0.424 \\ 
  GA & 0.083 \\ 
  GF\% & 0.705 \\ 
  xGF & 0.387 \\ 
  xGA & 0.196 \\ 
  xGF\% & 0.586 \\ 
  SCF & 0.387 \\ 
  SCA & 0.192 \\ 
  SCF\% & 0.585 \\ 
  HDCF & 0.399 \\ 
  HDCA & 0.196 \\ 
  HDCF\% & 0.551 \\ 
  HDGF & 0.442 \\ 
  HDGA & 0.093 \\ 
  HDGF\% & 0.614 \\ 
  On-Ice SH\% & 0.324 \\ 
  On-Ice SV\% & 0.419 \\ 
  PDO & 0.546 \\ 
  \bottomrule
\end{tabular}
\end{table}


Metrics such as GF\%, FF\%, SF\%, PDO, xGF\%, and HDGF\% show moderate to strong positive correlations with plus-minus (0.705, 0.554, 
0.569, 0.546, 0.586, and 0.614, respectively). This result aligns with expectations, as many of these metrics are related to goal-scoring 
chances and shot control, which contribute directly to team scoring and subsequently affect plus-minus. Higher correlations among these 
metrics indicate a similar variation pattern to plus-minus, suggesting that they capture overlapping aspects of a player's offensive and 
defensive performance.

\subsection{Ridge Regression Analysis of Offensive and Defensive Contributions}

Ridge regression was performed on three different sets of variables: combined metrics, offensive only, and defensive only. The combined 
metrics included GF\%, SF\%, PDO, Takeaways, and SCF\%, metrics that combine both offensive and defensive factors, similar to 
plus-minus.The offensive only metrics were CF, FF, SF, GF, SCF, HDCF, HDGF, GF\%, G, A, P/GP, CF\%, SCF\%, Rush Attempts, and On-Ice SH\%. 
The defensive only metrics were SA, GA, SCF, HDCA, HDGA, On-Ice SV\%, Hits, Shots Blocked, and Penalties Drawn. The results can be seen in the table below.

\begin{table}[tbp]
  \caption{Cross Validation Scores}
  \label{tab:rv2}
\centering
\begin{tabular}{rrr}
  \toprule
Both & Offensive & Defensive \\ 
  \midrule
  0.5696297 & 0.64692431 & 0.73265605 \\ 
  0.50125171 & 0.62667921 & 0.70862251 \\ 
  0.47617049 & 0.62263538 & 0.69467537 \\ 
  0.48944826 & 0.58484799 & 0.65315835 \\ 
  0.57586267 & 0.69145036 & 0.76239753 \\ 
  \bottomrule
\end{tabular}
\end{table}

The results in Table \ref{tab:rv2} reveal:

- **Combined Metrics**: A moderate cross-validation score around 0.53 suggests that these metrics explain a moderate amount of variance 
in plus-minus.

- **Offensive Metrics**: With an average score of 0.63, offensive metrics are predictive of plus-minus to a certain extent, indicating 
the offensive contributions captured by plus-minus.

- **Defensive Metrics**: The highest average score (around 0.71) suggests that defensive metrics have a more substantial impact on 
plus-minus, aligning with the fact that plus-minus also reflects defensive contributions.

Mixed Effects Model:

The results of the mixed effects model can be summarized in the table.

\begin{table}[tbp]
  \caption{Mixed Effects Model Results for Plus-Minus Prediction}
  \label{tab:mixed-effects}
  \centering
  \begin{tabular}{lrr}
    \toprule
    Predictor & Coefficient & Significance \\ 
    \midrule
    GF\% & 1.947 & 0.000 \\
    CF\% & 0.446 & 0.478 \\
    SCF\% & -0.109 & 0.848 \\
    Team Mean CF\% & -3.111 & 0.000 \\ 
    Team Mean GF\% & 0.846 & 0.000 \\
    Team SCF\% & 3.127 & 0.001 \\ 
    \bottomrule
\end{tabular}
\end{table}

The coefficient for individual Corsi For percentage is a small, positive effect on plus/minus, though it is not statistically significant 
(p = 0.478). This suggests that while Corsi percentage (CF\%) has a slight positive association with plus/minus, it may not have a 
substantial or reliable impact on explaining variance in plus/minus at the individual level. Similarly, Scoring Chances For percentage 
(SCF\%) also does not have a significant relationship with plus/minus (p = 0.848).The positive and highly significant coefficient
for individual GF\% (p < 0.000) indicates a strong and reliable positive relationship between goals-for percentage and plus/minus. 
Players with a higher individual GF\% are likely to have a higher plus/minus score, suggesting that goal-scoring and offensive contribution 
are important for explaining plus/minus. 

The coefficent for Team Mean CF\% is negative, statistically significant coefficient (p < 0.000) indicates that the team’s mean CF\% 
negatively impacts individual plus/minus. In teams with higher Corsi percentages, individual players might have lower plus/minus scores, 
possibly due to the distribution of possession-based contributions across the team. The positive coefficient of Team Mean GF\%, 
significant at p < 0.01, confirms that team-level GF\% positively influences individual plus/minus. This suggests that players benefit 
from being on teams that are generally good at scoring, supporting the notion that plus/minus partially reflects team-level offensive 
strength. Finally, the positive and significant coefficient for Team SCF (p=0.001) indicates that teams generating more scoring chances 
positively affect individual plus/minus. This highlights the role of team offensive strength in influencing this statistic.

These results confirm that plus/minus is heavily influenced by team-level performance metrics (e.g., Team Mean GF\% and Team SCF) 
and less so by individual possession metrics like CF\% and SCF\%. The significant relationship between GF\% and plus/minus supports 
the notion that offensive contributions and team success are key drivers of this statistic, reinforcing that plus/minus is a 
team-oriented measure rather than solely reflective of individual performance.


\subsection{Plus-Minus vs Corsi vs Fenwick}

Both Corsi and Fenwick were used as predictors in ridge regression with the variables GF\%, SCF\%, HDCF\%, PDO, On-Ice SH\%, On-Ice SV\%, G, A.
The results of the cross validation are shown in the table below

\begin{table}[tbp]
  \caption{Cross Validation Scores}
  \label{tab:rv}
\centering
\begin{tabular}{rrr}
  \toprule
Corsi & Fenwick \\ 
  \midrule
  0.93898296 & 0.95099819 \\ 
  0.90054897 & 0.93012992 \\ 
  0.93915158 & 0.9528202 \\ 
  0.92841222 & 0.95201592 \\ 
  0.9325304 & 0.95921829 \\ 
   \bottomrule
\end{tabular}
\end{table}

These values are notably better that the cross validation scores for plus-minus, regardless of which set of predictors. This indicates
that both Corsi and Fenwick are significantly better at predicting player effectiveness than plus-minus.


\section{Discussion}
\label{sec:disc}

Like the plus-minus statistic, this project does not come without its limitations. This analysis relies on specific datasets from 
Natural Stat Trick and includes only a selection of individual and team-based metrics. Other relevant factors like zone entries, exits, or 
additional situational metrics might provide further insights into player contributions but were not available in this dataset. The model 
does not capture all nuances of game context, such as player fatigue, line changes, and shifts against specific opponents, which can all 
influence a player’s plus/minus. The mixed-effects model approximates some of these factors with team-level random effects, but a more 
complex model might be required for a comprehensive understanding.

In this paper, the effectiveness of the plus-minus statistic was evaluated to see if it is an effective predictor of player contribution
in the National Hockey League. The plus minus does have some value as a predictor of offensive and defensive metrics. However, other
advanced metrics like Corsi and Fenwick perform much better. The plus-minus is also heavily influenced by a player's team, making it 
less valuable as an individual evaluator of player effectiveness. Overall, the NHL should consider phasing out the use of the plus-minus
statistic and consider more advanced metrics such as Corsi or Fenwick.

In order for fans, players, and sports management to have a better understanding of what make a player "effective", you have to 
summarize and quantify a large number of variables into one condensed number. That is, understandably, a tall order. While the plus-minus
lays the ground-work for getting a better understanding of player effectiveness, there is more at play. General managers of sports teams
often want to make the best team possible, getting players that will have a positive impact. Often times, people put a lot of emphasis on
direct offensive output, such as points, to determine a players contribution. But there is so much more to consider, such as defensive skills,
generating chances to score, etc, while still accounting for team effects. Many have tried to do this, but lack the interpretability and 
simplicity of the plus-minus. While plus-minus may not be the solution, the next step in the world of sports statistics is to find a way to 
quantify the concept of "player effectiveness" in a way that includes the simplicity of the plus-minus without its controversial nature.

\bibliography{refs}
\bibliographystyle{mcap}

\end{document}